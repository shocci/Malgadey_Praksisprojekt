\chapter{Erwartungen}
\label{cha:Erwartungen}

\section{Erwartete Artefakte}
\begin{itemize}
  \item \textbf{Physische Artefakte:} Testdrucke mit verschiedenen Support-Geometrien (FDM vs. SLA-inspiriert).  
  \item \textbf{Datensätze:} Tabellen mit Messwerten zu Maßhaltigkeit, Haftung, Oberflächenbeschaffenheit.  
  \item \textbf{Analyse-Skripte:} Python-Auswertungen zur grafischen Darstellung von Qualitätsmetriken.  
  \item \textbf{Workflow-Dokumentation:} Schritt-für-Schritt-Anleitung zur Umsetzung SLA-inspirierter Supports im FDM-Druck.  
  \item \textbf{Optimiertes Slicer-Profil:} Beispielhafte OrcaSlicer-Konfiguration mit den besten getesteten Parametern.
\end{itemize}

\section{Erwartete Ergebnisse}
Es wird erwartet, dass die Übertragung von SLA-Support-Strukturprinzipien auf den FDM-Druck praktikabel ist, sofern geometrische Anpassungen vorgenommen werden.  
Insbesondere kleinere Kontaktpunkte und dünnere Stützsäulen könnten zu einer verbesserten Oberflächenqualität und leichteren Entfernung führen.  
Gleichzeitig ist anzunehmen, dass zu filigrane Strukturen im FDM aufgrund des schichtweisen Materialauftrags an Stabilität verlieren.  
Das Projekt soll somit einen Beitrag zur Optimierung von Support-Strukturen im FDM leisten und eine Grundlage für zukünftige Slicer-Integrationen schaffen.