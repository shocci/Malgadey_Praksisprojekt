\chapter{Methodisches Vorgehen}
\label{cha:Methodisches Vorgehen}

Das Projekt folgt einem experimentellen Ansatz, bei dem SLA-inspirierte Support-Strukturen für das FDM-Verfahren modelliert, gedruckt und mit bestehenden FDM-Supporttypen verglichen werden.  

\section{Versuchsaufbau}

Für die Tests werden ein \textit{Bambu Lab A1 Mini} und ein \textit{Prusa MK3 mit E3D Revo-Upgradekit} verwendet.  
Diese Systeme repräsentieren typische FDM-Drucker für Einsteiger- und Semi-Profi-Anwendungen.  
Als Filamente werden PLA und PETG eingesetzt, da sie unterschiedliche Haftungs- und Schrumpfverhalten zeigen \cite{Kristiawan2021}.  

\section{Untersuchungsparameter}

\begin{itemize}
  \item Strukturvarianten: traditionelle FDM-Supports vs. SLA-inspirierte Supportformen (Base, Columns, Contact Points)  
  \item Geometrieparameter: Säulendurchmesser, Säulenabstand, Kontaktpunktgröße, Basisdicke  
  \item Prozessparameter: Druckbett- und Düsentemperatur, Schichthöhe, Druckgeschwindigkeit  
  \item Evaluationskriterien: Haftung, Maßhaltigkeit, Oberflächenqualität, Nachbearbeitungsaufwand
\end{itemize}

\section{Datenerhebung}

Jede Strukturvariante wird mehrfach unter konstanten Bedingungen gedruckt.  
Die Ergebnisse werden quantitativ (Messwerte, Warping, Abweichungen) und qualitativ (Bildanalyse, visuelle Bewertung der Oberflächen) erfasst.  
Zur Auswertung werden statistische Verfahren eingesetzt, um signifikante Unterschiede zwischen den Strukturtypen zu bestimmen \cite{FFFPolymerReview}.
