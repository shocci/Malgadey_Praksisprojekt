\chapter{Methodik / Vorgehensweise}
\label{cha:Methodik / Vorgehensweise}

\section{Theoretische Recherche}
FDM-Prozess, Haftmechanismen (Adhäsion, Reibung), Materialverhalten (PLA, PETG), Einfluss der Druckbettoberfläche.


\section{Versuchsaufbau}
\begin{itemize}
\item Drucker: z.~B. Creality Ender~3 / Prusa MINI (Einsteiger-Segment)
\item Materialien: PLA, PETG (gleiche Herstellerlinie, Konstanz der Charge)
\item Testgeometrien: Kalibrierplättchen, Brim-Sensitivitätsprobe, kleine Stützflächen
\item Konstante Randbedingungen: Raumtemp., Luftzug, Düsen-/Bettreinigung
\end{itemize}


\section{Parameterstudien}
\begin{itemize}
\item \textbf{Brim}: Breite (mm), Linienanzahl, Abstand (Gap), Kontaktstrategie
\item \textbf{Prozess}: Druckbett-Temperatur, erste-Layer-Geschwindigkeit, erste-Layer-Höhe
\item \textbf{Material}: PLA vs. PETG
\end{itemize}


\section{Messung}
\begin{itemize}
\item Haftung: \emph{Peel-/Shear-Test} (standardisierte Abzieh-/Scherversuche)
\item Maßhaltigkeit \/ Verzug: Messschieber, planare Bildauswertung
\item Oberflächenqualität: visuelle Skala, Makroaufnahmen
\end{itemize}


\section{Auswertung}
Varianzanalysen (ANOVA) bzw. nichtparametrische Tests; Diagramme und Tabellen; Ableitung von Optima/Trade-offs.