\chapter{Methodisches Vorgehen}
\label{cha:Methodisches Vorgehen}

Das methodische Vorgehen orientiert sich an einem experimentellen Ansatz, bei dem der Einfluss verschiedener \textit{Brim-Support-Parameter} auf die Druckqualität systematisch untersucht wird. Der Fokus liegt dabei auf Faktoren, die für Einsteiger- und Modellbauanwendungen besonders relevant sind, wie etwa Linienanzahl, Breite, Materialtyp und Drucktemperatur \cite{Kristiawan2021, FFFPolymerReview}.  

\subsection{Versuchsaufbau}

Die Druckversuche werden mit zwei unterschiedlichen FDM-Drucksystemen durchgeführt: einem \textit{Bambu Lab A1 Mini} und einem \textit{Prusa MK3} mit \textit{E3D Revo-Upgradekit}. Diese Geräte wurden ausgewählt, da sie repräsentativ für den Einsteiger- und Semi-Profi-Bereich stehen und eine gute Vergleichbarkeit zwischen unterschiedlichen Drucksystemen ermöglichen.  

Als Druckmaterialien kommen standardisierte Filamente wie \textit{PLA} und \textit{PETG} zum Einsatz, da diese typischerweise von Einsteigern verwendet werden und sich durch unterschiedliche Haftungs- und Schrumpfverhalten auszeichnen \cite{FDMReview2021}. Die Druckobjekte umfassen sowohl \textit{Miniaturen mit filigranen Details} als auch \textit{mehrteilige Modellbau- oder Skulpturenelemente}. Ergänzend werden \textit{Kalibrierobjekte} eingesetzt, um Maßhaltigkeit und Oberflächenqualität vergleichend zu bewerten.  

\subsection{Versuchsparameter und Design}

Die Untersuchung erfolgt auf Grundlage eines faktoriellen Versuchsplans, bei dem die wichtigsten Parameter systematisch variiert werden.  
Zu den unabhängigen Variablen zählen:
\begin{itemize}
  \item Brim-Linienanzahl (z.\,B. 2, 4, 6, 8)
  \item Brim-Breite (z.\,B. 2\,mm, 4\,mm, 8\,mm)
  \item Materialtyp (PLA)
  \item Druckbett- und Düsentemperatur (z.\,B. ±5 °C Variation)
  \item Support-Typ (Brim, Standard, organisch)
\end{itemize}

Jede Parameterkombination wird mehrfach gedruckt, um reproduzierbare Ergebnisse zu gewährleisten. Für jede Druckserie werden Haftung, Maßhaltigkeit, Oberflächenqualität und Nachbearbeitungsaufwand dokumentiert.  

\subsection{Datenerhebung und Bewertung}

Die Bewertung der Druckergebnisse erfolgt anhand quantitativer und qualitativer Kriterien.  
Zu den quantitativen Messgrößen zählen:
\begin{itemize}
  \item Maßabweichungen in X-, Y- und Z-Richtung (Messschieber oder optische Vermessung)
  \item Auftreten von Warping oder Ablösungen
  \item Druckzeit und Materialverbrauch
\end{itemize}

Ergänzend wird eine qualitative Beurteilung vorgenommen, die visuelle Merkmale wie Kantenschärfe, Detailtreue und Oberflächenfehler berücksichtigt \cite{FFFPolymerReview}. Die Ergebnisse werden statistisch ausgewertet, um signifikante Unterschiede zwischen den Parametern zu identifizieren.  

\subsection{Auswertung und Dokumentation}

Die gesammelten Daten werden in tabellarischer Form erfasst und anschließend mit geeigneten Auswertungsverfahren (z.\,B. Varianzanalyse oder Mittelwertvergleich) untersucht.  
Bilder und Mikroskopaufnahmen dienen der visuellen Dokumentation der Druckqualität. Auf Grundlage dieser Ergebnisse werden Handlungsempfehlungen formuliert und in einem Workflow zusammengeführt, der die Brim-Nutzung in unterschiedlichen Anwendungsszenarien beschreibt.  

Der Workflow soll abschließend in seiner Anwendbarkeit evaluiert und hinsichtlich einer möglichen Integration in Open-Source-Slicer wie \textit{OrcaSlicer} oder Modellierungssoftware wie \textit{Blender} bewertet werden. Dadurch soll eine praxisnahe und reproduzierbare Vorgehensweise geschaffen werden, die Anwenderinnen und Anwender bei der effektiven Nutzung von Brim-Support-Strukturen unterstützt.
