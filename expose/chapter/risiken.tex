\chapter{Risiken, Grenzen und ethische Aspekte}
\label{cha:Risiken, Grenzen und ethische Aspekte}

\subsection{Risiken und Grenzen}

Trotz sorgfältiger Planung bestehen bei der Durchführung des Projekts verschiedene potenzielle Risiken und Grenzen, die berücksichtigt werden müssen:

\begin{itemize}
  \item \textbf{Technische Risiken:} 
  Fehldrucke, Materialschwankungen oder unzureichende Kalibrierung der Drucker können zu Messabweichungen führen.  
  Schwankungen in der Raumtemperatur oder in der Filamentqualität könnten die Ergebnisse zusätzlich beeinflussen und müssen daher kontrolliert dokumentiert werden \cite{Kristiawan2021}.
  
  \item \textbf{Methodische Grenzen:} 
  Die Untersuchung beschränkt sich auf eine ausgewählte Anzahl von Parametern (z.\,B. Linienanzahl, Breite, Temperatur) und auf zwei Drucksysteme.  
  Daher können die Ergebnisse nicht ohne Weiteres auf andere Geräte, Materialien oder Drucktechnologien übertragen werden \cite{FFFPolymerReview}.
  
  \item \textbf{Zeitliche Einschränkungen:} 
  Der verfügbare Projektzeitraum (Ende November bis März) begrenzt die Anzahl möglicher Wiederholungen und Variationen.  
  Eine vollständige statistische Absicherung aller Parameterkombinationen kann daher nur in ausgewählten Fällen erfolgen.
  
  \item \textbf{Subjektive Beurteilung:} 
  Qualitative Bewertungen, etwa der Oberflächenqualität oder Detailtreue, enthalten einen subjektiven Anteil.  
  Um diesen zu minimieren, werden standardisierte Kriterien und fotografische Dokumentation eingesetzt.
\end{itemize}

\subsection{Ethische Aspekte}

Das Projekt unterliegt keinen direkten ethischen Risiken im Sinne von Datenschutz, Umwelteinflüssen oder Auswirkungen auf Personen.  
Trotzdem werden grundlegende wissenschaftliche und ethische Prinzipien beachtet:

\begin{itemize}
  \item \textbf{Datenintegrität:} 
  Alle Messergebnisse werden unverändert dokumentiert und transparent ausgewertet.  
  Etwaige Ausreißer oder Messfehler werden nachvollziehbar angegeben und nicht nachträglich entfernt.
  
  \item \textbf{Urheberrecht:} 
  Bei der Verwendung von 3D-Modellen wird darauf geachtet, ausschließlich selbst erstellte oder lizenzfreie Modelle (z.\,B. unter Creative Commons) zu nutzen.  
  Quellen und Ersteller werden in der Dokumentation entsprechend genannt.
  
  \item \textbf{Nachhaltigkeit:} 
  Der Materialverbrauch wird auf das notwendige Minimum beschränkt. Fehl- oder Testdrucke werden, sofern möglich, recycelt oder für Kalibrierzwecke wiederverwendet.
\end{itemize}

\subsection{Zusammenfassung}

Die genannten Risiken und Grenzen werden im Rahmen der Projektplanung berücksichtigt und durch präventive Maßnahmen (z.\,B. Kalibrierkontrollen, Wiederholungsdrucke, standardisierte Dokumentation) reduziert.  
Ethische Grundsätze wie Transparenz, Nachvollziehbarkeit und verantwortungsvoller Umgang mit Ressourcen bilden die Grundlage der gesamten Arbeit.
