\chapter{Forschungsfragen}
\label{cha:Forschungsfragen}

Ausgehend von der beschriebenen Zielsetzung ergeben sich für das Projekt die folgenden zentralen Forschungsfragen:

\begin{enumerate}
  \item In welchem Umfang beeinflussen \textit{Brim-Support-Strukturen} die \textit{Detailgenauigkeit} und \textit{Oberflächenqualität} von FDM-gedruckten Objekten?
  \item Welche Unterschiede bestehen zwischen \textit{Brim-, Standard-} und \textit{organischen Support-Strukturen} in Bezug auf Haftung, Maßhaltigkeit und Nachbearbeitungsaufwand \cite{Facfox2023, JLC3DP2023}?
  \item Welche Parameterkombinationen (z.\,B. Linienanzahl, Breite, Material, Temperatur) führen zu reproduzierbar guten Druckergebnissen bei filigranen Objekten wie Miniaturen oder Skulpturen \cite{Kristiawan2021, FFFPolymerReview}?
  \item Inwiefern unterscheiden sich die Ergebnisse zwischen unterschiedlichen Drucksystemen, insbesondere zwischen Einsteigergeräten (z.\,B. \textit{Bambu Lab A1 Mini}) und Semi-Profi-Druckern (z.\,B. \textit{Prusa MK3 mit E3D Revo}) \cite{FDMReview2021}?
  \item Wie kann auf Basis der Untersuchung ein \textit{strukturierter Workflow} entwickelt werden, der den Einsatz von Brim-Support-Strukturen in praktischen Druckprozessen standardisiert und als Grundlage für eine mögliche Software-Integration dient?
\end{enumerate}

Diese Fragen bilden den Leitfaden für die experimentelle Untersuchung und dienen als Grundlage für die Auswertung und Diskussion der Ergebnisse.