\chapter*{Kurzfassung}

Das \textit{Fused-Deposition-Modeling-Verfahren (FDM)} ist ein verbreitetes Verfahren der additiven Fertigung, bei dem thermoplastisches Filament schichtweise aufgetragen und verfestigt wird.  
Es findet Anwendung in einer Vielzahl von Bereichen, insbesondere im Umfeld von Einsteiger- und Desktop-3D-Drucksystemen.

Im Rahmen dieses Projekts wird untersucht, in welchem Umfang die aus dem \textit{Stereolithografie-Verfahren (SLA)} bekannten Support-Strukturprinzipien – bestehend aus einer \textit{Base} (Fußplatte), \textit{Support Columns} (Stützsäulen) und \textit{Contact Points} (Kontaktpunkte) – auf das FDM-Druckverfahren übertragbar sind.  
Ziel ist es, zu analysieren, wie sich diese SLA-inspirierten Strukturen im Vergleich zu herkömmlichen FDM-Supporttypen hinsichtlich Haftung, Maßhaltigkeit, Detailgenauigkeit und Nachbearbeitungsaufwand verhalten.

Der Fokus liegt dabei auf detailreichen Druckobjekten wie Miniaturen, Figuren und kleinformatigen Skulpturen, bei denen die Oberflächenqualität und Präzision eine zentrale Rolle spielen.  
Die Experimente werden unter praxisnahen Bedingungen durchgeführt, die sich an Einsteiger- und Hobbyanwendungen orientieren.  
Zum Einsatz kommen ein \textit{Bambu Lab A1 Mini} sowie ergänzend ein \textit{Prusa MK3 mit E3D-Revo-Upgradekit}, um unterschiedliche Systeme und Druckparameter miteinander vergleichen zu können.

Ziel des Projekts ist die Entwicklung eines strukturierten Workflows, der eine geordnete und reproduzierbare Vorgehensweise zur Nutzung von SLA-inspirierten Support-Strukturen im FDM-Druck ermöglicht.  
Dieser Workflow soll Anwender:innen eine einfache Möglichkeit bieten, angepasste Support-Designs für detailreiche Modelle zu erstellen und sowohl bei vorgestützten (presupported) als auch bei selbst gestützten (supported) Modellen anzuwenden.  

Langfristig soll das Projekt eine Grundlage für die Integration entsprechender Funktionen in gängige Open-Source-Software wie \textit{Blender} oder \textit{OrcaSlicer} schaffen, um eine benutzerfreundliche und nachvollziehbare Implementierung solcher Strukturen im FDM-Druckprozess zu ermöglichen.
