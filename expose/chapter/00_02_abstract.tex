\chapter*{Kurzfassung}
Das Fused-Deposition-Modeling-Verfahren (FDM) ist ein Verfahren der additiven Fertigung, bei dem thermoplastisches Filament schichtweise aufgetragen und verfestigt wird. Es wird unter anderem im Bereich des Modellbaus und von Einsteiger-3D-Drucksystemen eingesetzt.

In diesem Projekt wird untersucht, in welchem Umfang Brim-Support-Strukturen die Detailgenauigkeit, Maßhaltigkeit und Oberflächenbeschaffenheit von FDM-gedruckten Objekten beeinflussen. Der Schwerpunkt liegt auf detailreichen Anwendungen wie Miniaturen, Modellbauteilen und mehrteiligen 3D-Skulpturen.

Untersucht wird der Unterschied zwischen Brim-Support-Strukturen und Standard- sowie organischen Support-Typen hinsichtlich Haftung, Druckqualität und Nachbearbeitungsaufwand. Die Versuche erfolgen unter Bedingungen, die sich an den typischen Rahmenbedingungen von Einsteiger- und Hobbyanwendungen orientieren. Hierfür werden ein Bambu Lab A1 Mini sowie ergänzend ein Prusa MK3 mit E3D-Revo-Upgradekit verwendet.

Ziel des Projekts ist die Entwicklung eines strukturierten Workflows, der eine geordnete und reproduzierbare Vorgehensweise zur Nutzung von Brim-Support-Strukturen bietet. Dieser Workflow soll es ermöglichen, Brim-Supports sowohl bei vorgesupporteten Modellen als auch bei eigenständig supporteten Modellen effizient einzusetzen.\\
Langfristig soll das Projekt eine Grundlage für eine Integration entsprechender Funktionen in gängige Open-Source-Software wie Blender oder OrcaSlicer schaffen, um Anwendern eine einfache und nachvollziehbare Implementierung zu ermöglichen.