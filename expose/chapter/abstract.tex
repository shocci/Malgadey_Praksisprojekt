\chapter*{Kurzfassung}
Der 3D-Druck nach dem Fused-Deposition-Modeling-Verfahren (FDM) ist eine zentrale Technologie im Bereich der additiven Fertigung, insbesondere im Einsteiger- und Modellbaukontext.
Ein häufiges Problem besteht in der unzureichenden Haftung zwischen Druckbett und Bauteil, was zu Verzug und Druckfehlern führt.
Dieses Projekt untersucht systematisch die Wirkung von Brimsupport-Strukturen auf die Haftung, Maßhaltigkeit und Druckqualität bei Einsteiger-FDM-Druckern.
Dazu werden unterschiedliche Parameterkombinationen (Linienanzahl, Breite, Material, Temperatur) experimentell getestet und ausgewertet.
Ziel ist es, optimierte Brim-Einstellungen zu identifizieren und ein standardisiertes Slicer-Profil zu entwickeln, das Einsteigern den Druckprozess erleichtert und die Fehlerrate reduziert.