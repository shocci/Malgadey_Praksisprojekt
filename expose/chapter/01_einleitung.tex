\chapter{Einleitung / Ausgangslage und Motivation}
\label{cha:Einleitung / Ausgangslage und Motivation}

Support-Strukturen sind ein wesentlicher Bestandteil vieler 3D-Druckverfahren, da sie während des Druckprozesses Überhänge und freistehende Geometrien stabilisieren.  
Im SLA-Verfahren bestehen Support-Strukturen typischerweise aus drei funktionalen Komponenten:\\
einer \textit{Base} (Fußplatte), die an der Bauplattform haftet,\\
\textit{Support Columns} (Stützsäulen), die das Modell tragen,\\
und \textit{Contact Points} (Kontaktpunkte), die das Modell nur minimal berühren, um eine einfache Entfernung nach dem Druck zu ermöglichen \cite{FormlabsSupports}.  

Diese dreiteilige Struktur bietet beim SLA-Druck eine gute Balance zwischen Stabilität, Materialeffizienz und leichter Nachbearbeitung.  
Im Gegensatz dazu nutzt das FDM-Verfahren andere Supportstrategien, wie Raster- oder organische Supports, die primär auf dem Prinzip der Materialüberlagerung basieren \cite{JiangSupportReview, Kristiawan2021}.\\
Bisher existieren jedoch kaum Untersuchungen dazu, inwieweit die aus SLA bekannten Strukturprinzipien (Base, Stützsäulen, Kontaktpunkte) auf FDM-Druckverfahren übertragbar sind.  

Die persönliche Motivation für dieses Projekt ergibt sich aus der praktischen Arbeit mit dem FDM-Druck detailreicher Modelle – etwa Miniaturen, Figuren oder kleinformatiger Skulpturen.  
Hier treten regelmäßig Probleme auf, die in SLA-Drucksystemen durch optimierte Support-Geometrien bereits reduziert werden konnten.  
Aus dieser Beobachtung entstand die Forschungsfrage, ob die Prinzipien der SLA-Support-Strukturen auch im FDM-Druck nutzbar sind und welche Anpassungen dafür erforderlich wären.  
