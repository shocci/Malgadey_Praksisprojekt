\chapter{Zielsetzung}
\label{cha:Zielsetzung}

Ziel dieses Projekts ist es, den Einfluss von \textit{Brim-Support-Strukturen} auf die \textit{Detailgenauigkeit}, \textit{Maßhaltigkeit} und \textit{Oberflächenqualität} von FDM-gedruckten Objekten systematisch zu untersuchen. Dabei soll ermittelt werden, in welchem Umfang unterschiedliche Brim-Parameter (z.\,B. Linienanzahl, Breite, Material und Drucktemperatur) die Druckqualität bei detailreichen Objekten wie Miniaturen, Modellbauteilen oder mehrteiligen Skulpturen beeinflussen \cite{Kristiawan2021, FDMReview2021, FFFPolymerReview}.  

Ein weiterer Schwerpunkt liegt auf dem Vergleich zwischen Brim-Support-Strukturen und herkömmlichen \textit{Standard-} sowie \textit{organischen Support-Typen}. Ziel ist es, die jeweiligen Vor- und Nachteile hinsichtlich Haftung, Stabilität, Materialverbrauch und Nachbearbeitungsaufwand zu identifizieren \cite{Facfox2023, JLC3DP2023}.  

Auf Grundlage der experimentellen Ergebnisse soll ein \textit{strukturierter Workflow} entwickelt werden, der eine geordnete und reproduzierbare Vorgehensweise für den Einsatz von Brim-Support-Strukturen bietet. Dieser Workflow soll sowohl bei \textit{vorgesupporteten Modellen} als auch bei \textit{eigenständig supporteten Modellen} anwendbar sein und Anwenderinnen und Anwendern als Orientierung für die praktische Umsetzung dienen.  

Langfristig soll das Projekt damit die Grundlage für eine mögliche \textit{Software-Integration} entsprechender Funktionen in Open-Source-Anwendungen wie \textit{Blender} oder \textit{OrcaSlicer} schaffen. Dadurch könnten Brim-Strukturen künftig automatisiert oder teilautomatisiert im Slicer-Workflow eingesetzt werden, um die Druckqualität bei Einsteiger- und Modellbauanwendungen gezielt zu verbessern.