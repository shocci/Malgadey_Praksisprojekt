\chapter{Zielsetzung}
\label{cha:Zielsetzung}

Ziel des Projekts ist es, die Übertragbarkeit der im SLA-Druck verwendeten Support-Strukturprinzipien – \textit{Base}, \textit{Support Columns} und \textit{Contact Points} – auf das FDM-Verfahren systematisch zu untersuchen, insbesondere im Kontext detailreicher 3D-Modelle und Miniaturen.  
Dabei soll analysiert werden:

\begin{itemize}
  \item welche geometrischen und materialtechnischen Anpassungen notwendig sind, um SLA-ähnliche Strukturen im FDM-Druck zu realisieren,
  \item wie sich diese Strukturen im Vergleich zu herkömmlichen FDM-Supports hinsichtlich Haftung, Stabilität, Nachbearbeitbarkeit und Druckqualität verhalten,
  \item und ob sich daraus ein Workflow ableiten lässt, der diese Strukturen in FDM-Slicern nutzbar macht.
\end{itemize}

Langfristig soll auf Basis der Ergebnisse ein standardisierter Workflow entstehen, der für FDM-Nutzer:innen eine klare Vorgehensweise bietet, um optimierte Stützstrukturen zu entwerfen oder automatisiert zu generieren.  
Darüber hinaus wird geprüft, inwiefern sich eine Implementierung in Open-Source-Slicer wie \textit{OrcaSlicer} oder Modellierungssoftware wie \textit{Blender} anbietet.
