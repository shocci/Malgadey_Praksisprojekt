\chapter{Zeitplan}
\label{cha:Zeitplan}

Das Projekt ist auf einen Zeitraum von etwa vier Monaten ausgelegt und soll voraussichtlich \textbf{Ende November} beginnen und bis \textbf{März} abgeschlossen sein.  
Der Zeitplan gliedert sich in mehrere Phasen, die aufeinander aufbauen und sowohl die theoretische Vorbereitung als auch die experimentelle Durchführung und Auswertung umfassen.

\begin{table}[h!]
\centering
\begin{tabular}{p{4cm} p{9cm}}
\textbf{Zeitraum} & \textbf{Arbeitsschritte und Inhalte} \\
\hline
Ende November – Mitte Dezember & 
Literaturrecherche zu FDM, Brim- und Support-Strukturen; Einarbeitung in theoretische Grundlagen und Dokumentation relevanter Quellen. 
Festlegung der Versuchsparameter und Erstellung eines detaillierten Versuchsplans. \\[6pt]

Mitte Dezember – Anfang Januar & 
Vorbereitung des Versuchsaufbaus, Kalibrierung der Drucksysteme (\textit{Bambu Lab A1 Mini} und \textit{Prusa MK3}). 
Testdrucke zur Validierung der gewählten Parameter. \\[6pt]

Januar & 
Durchführung der Hauptversuchsreihen mit Variation der Brim-Parameter (Linienanzahl, Breite, Material, Temperatur). 
Erhebung und Dokumentation der Messdaten (Haftung, Maßhaltigkeit, Detailqualität). \\[6pt]

Ende Januar – Mitte Februar & 
Auswertung der erhobenen Daten, qualitative und quantitative Analyse. 
Vergleich der Ergebnisse zwischen Drucksystemen und Support-Typen. \\[6pt]

Mitte Februar – Anfang März & 
Entwicklung und Dokumentation des strukturierten Workflows. 
Ableitung praxisorientierter Empfehlungen und Bewertung einer möglichen Software-Integration in \textit{Blender} oder \textit{OrcaSlicer}. \\[6pt]

März & 
Abschlussphase: Zusammenführung der Ergebnisse, Endredaktion und Erstellung des Abschlussdokuments. 
Vorbereitung der Präsentation oder Verteidigung des Projekts. \\
\hline
\end{tabular}
\caption{Geplanter Zeitrahmen für das Projekt zur Untersuchung von Brim-Support-Strukturen im FDM-Druck.}
\end{table}

Der Zeitplan berücksichtigt sowohl Pufferzeiten für unvorhergesehene Verzögerungen (z.\,B. Fehldrucke oder technische Probleme) als auch ausreichend Zeit für die Auswertung und Aufbereitung der Ergebnisse.
