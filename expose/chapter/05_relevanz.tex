\chapter{Relevanz und erwarteter Nutzen}
\label{cha:Relevanz und erwarteter Nutzen}

Das Projekt besitzt wissenschaftliche und praktische Relevanz für den Bereich der additiven Fertigung.  
Wissenschaftlich trägt es dazu bei, den Einfluss alternativer Support-Geometrien auf die Druckqualität im FDM-Verfahren zu verstehen.  
Praktisch kann es zu einer Optimierung der Druckergebnisse beitragen, insbesondere bei filigranen 3D-Modellen, Miniaturen und anderen detailreichen Objekten.\\

Die entwickelten Erkenntnisse sollen dazu dienen:
\begin{itemize}
  \item den Materialverbrauch und Nachbearbeitungsaufwand zu reduzieren,
  \item reproduzierbare Workflows für Einsteiger und Maker bereitzustellen,
  \item und langfristig neue Slicer-Funktionen zur automatisierten Generierung von SLA-inspirierten Stützstrukturen zu ermöglichen.
\end{itemize}
