\chapter{Ausgangslage und Motivation}
\label{cha:Ausgangslage und Motivation}

Der 3D-Druck mittels Fused Deposition Modeling (FDM) hat sich als zugängliche und kostengünstige Fertigungsmethode etabliert – insbesondere im Bereich des Modellbaus und bei Einsteigern in die additive Fertigung.
Ein häufiges Problem besteht jedoch in unzureichender Haftung zwischen Bauteil und Druckbett, was zu Verzug, Ablösung oder Druckfehlern führt.

Brimsupport-Strukturen – flache, umlaufende Materialränder – dienen der Verbesserung der Haftung und Stabilität während des Druckprozesses.
Trotz ihrer Relevanz gibt es kaum systematische Untersuchungen, wie sich verschiedene Brim-Parameter (z. B. Linienanzahl, Breite, Abstand) auf die Druckqualität auswirken – insbesondere auf Einsteiger-Druckern mit begrenzter Präzision.

Das Projekt zielt darauf ab, diese Parameter technisch zu analysieren, zu vergleichen und Empfehlungen für Einsteiger zu entwickeln.