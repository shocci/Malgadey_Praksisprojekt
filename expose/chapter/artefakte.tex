\chapter{Erwartete Artefakte}
\label{cha:Erwartete Artefakte}

Im Verlauf des Projekts werden verschiedene Artefakte entstehen, die sowohl die experimentellen Ergebnisse als auch die entwickelten Arbeitsabläufe dokumentieren.  
Diese Artefakte lassen sich in drei Hauptkategorien unterteilen: physische Druckergebnisse, digitale Datensätze und dokumentarische bzw. konzeptionelle Ergebnisse.

\subsection{Physische Artefakte}

\begin{itemize}
  \item \textbf{Testdrucke und Vergleichsobjekte:} 
  Eine Reihe von Miniaturen, Modellteilen und Kalibrierobjekten, die mit unterschiedlichen Brim- und Support-Einstellungen gedruckt wurden.  
  Diese Drucke dienen als Grundlage für die Bewertung von Haftung, Maßhaltigkeit und Detailqualität.
  
  \item \textbf{Referenzmodelle:} 
  Ausgewählte Druckobjekte, die die jeweils besten und schlechtesten Parameterkombinationen darstellen, werden als Referenzbeispiele archiviert.  
  Sie können zur visuellen Demonstration der Ergebnisse oder für Folgestudien herangezogen werden.
\end{itemize}

\subsection{Digitale Artefakte}

\begin{itemize}
  \item \textbf{Datensätze und Messprotokolle:} 
  Tabellarische Aufzeichnungen der Messwerte (z.\,B. Maßabweichungen, Materialverbrauch, Haftungsergebnisse) sowie begleitende Bilddokumentationen der Druckoberflächen.  
  Diese Daten werden in strukturierter Form gespeichert und dienen als Grundlage für statistische Auswertungen und Reproduzierbarkeit.
  
  \item \textbf{Analyse- und Auswertungsskripte:} 
  Python-basierte Auswertungsdateien (\textit{pandas}, \textit{matplotlib}) zur automatisierten Berechnung und Visualisierung von Kennwerten.  
  Diese Skripte ermöglichen eine spätere Wiederverwendung oder Erweiterung im Rahmen weiterer Untersuchungen.
\end{itemize}

\subsection{Konzeptionelle und dokumentarische Artefakte}

\begin{itemize}
  \item \textbf{Workflow-Dokumentation:} 
  Eine schriftliche und grafische Darstellung des entwickelten Workflows, die den gesamten Ablauf von der Vorbereitung über den Druck bis zur Nachbearbeitung beschreibt.  
  Diese Dokumentation bildet die zentrale Ergebnisdarstellung des Projekts.
  
  \item \textbf{Optimiertes Slicer-Profil:} 
  Ein konfiguriertes \textit{OrcaSlicer}-Profil mit optimierten Brim-Parametern (z.\,B. Linienanzahl, Breite, Temperatur).  
  Das Profil soll als reproduzierbare Grundlage für Einsteiger dienen und kann als Beispiel für eine mögliche Software-Integration genutzt werden.
  
  \item \textbf{Abschlussbericht:} 
  Eine zusammenfassende Projektarbeit mit Dokumentation der Vorgehensweise, Ergebnisse, Auswertungen und Schlussfolgerungen.  
  Der Bericht dient zugleich als Grundlage für eine spätere Veröffentlichung oder Erweiterung des Projekts.
\end{itemize}

\subsection{Zusammenfassung}

Die erwarteten Artefakte ermöglichen eine umfassende Bewertung der Brim-Support-Strukturen sowohl auf praktischer als auch auf analytischer Ebene.  
Sie schaffen damit eine Grundlage für die Übertragbarkeit der Ergebnisse auf zukünftige Anwendungen und potenzielle Software-Integrationen im Bereich der additiven Fertigung.
