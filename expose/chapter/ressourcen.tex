\chapter{Ressourcen / Equipment}
\label{cha:Ressourcen / Equipment}

Für die Durchführung des Projekts werden sowohl technische Geräte als auch Software- und Verbrauchsmaterialien benötigt.  
Der Ressourceneinsatz orientiert sich an typischen Anforderungen für experimentelle Untersuchungen im Bereich des Fused-Deposition-Modeling (FDM).

\subsection{Hardware}

\begin{itemize}
  \item \textbf{3D-Drucker:}
    \begin{itemize}
      \item \textit{Bambu Lab A1 Mini} – als Einsteigergerät für praxisnahe Testbedingungen.
      \item \textit{Prusa MK3} mit \textit{E3D Revo-Upgradekit} – als Vergleichssystem mit erweiterter Düsen- und Temperaturkontrolle.
    \end{itemize}

  \item \textbf{Mess- und Prüfmittel:}
    \begin{itemize}
      \item Digitaler Messschieber zur Erfassung von Maßhaltigkeit.
      \item Präzisionswaage zur Ermittlung des Materialverbrauchs.
      \item Dokumentationskamera oder Mikroskopaufsatz für Detailaufnahmen der Oberflächenqualität.
    \end{itemize}
  
  \item \textbf{Sonstiges Zubehör:}
    \begin{itemize}
      \item Heizbett mit justierbarer Temperatursteuerung.
      \item Werkzeuge zur Nachbearbeitung (z.\,B. Spachtel, Pinzetten, Schleifmittel).
      \item Kalibrierhilfen und Druckbett-Reinigungsmaterialien.
    \end{itemize}
\end{itemize}

\subsection{Materialien}

\begin{itemize}
  \item \textbf{Filamente:} PLA und PETG in standardisierten Farben zur besseren Sichtbarkeit von Oberflächenfehlern.
  \item \textbf{Haftmittel:} Druckbettkleber, PEI-Platte oder Klebestift zur standardisierten Haftungsbedingung.
  \item \textbf{Testobjekte:} Miniaturen und 3D-Skulpturen (vorgesupportet und eigenständig supportet) sowie Kalibrierobjekte zur Vergleichsmessung.
\end{itemize}

\subsection{Software}

\begin{itemize}
  \item \textbf{Slicer:} \textit{OrcaSlicer} (Open Source) für die Erstellung und Anpassung der Brim-Parameter.
  \item \textbf{3D-Modellierungssoftware:} \textit{Blender} zur Bearbeitung und Vorbereitung der Druckobjekte.
  \item \textbf{Dokumentation und Analyse:} Tabellenkalkulationssoftware (z.\,B. \textit{LibreOffice Calc} oder \textit{Microsoft Excel}) sowie Bildauswertungswerkzeuge zur Analyse von Oberflächenstrukturen.
  \item \textbf{Statistische Auswertung:} Python (Bibliotheken: \textit{NumPy}, \textit{pandas}, \textit{matplotlib}) oder \textit{R} für Datenanalyse und grafische Darstellung.
\end{itemize}

\subsection{Ressourcenplanung}

Alle genannten Geräte und Materialien sind im Rahmen des Projekts verfügbar oder können durch vorhandene Laborausstattung bereitgestellt werden.  
Für Software und Auswertungstools werden ausschließlich frei verfügbare oder Open-Source-Lösungen verwendet, um die Reproduzierbarkeit und Nachvollziehbarkeit der Ergebnisse zu gewährleisten.
