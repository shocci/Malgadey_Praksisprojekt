\chapter{Relevanz und erwarteter Nutzen}
\label{cha:Relevanz und erwarteter Nutzen}

Das Projekt besitzt sowohl wissenschaftliche als auch anwendungsorientierte Relevanz im Bereich der additiven Fertigung.  
Während der 3D-Druck im Fused-Deposition-Modeling-Verfahren (FDM) weit verbreitet ist, bestehen weiterhin Unsicherheiten hinsichtlich der optimalen Nutzung von Support- und Haftstrukturen, insbesondere im Zusammenhang mit filigranen Objekten \cite{Kristiawan2021, FDMReview2021}.  
Bisherige Untersuchungen konzentrieren sich vor allem auf mechanische Eigenschaften und Prozessoptimierung, während der Einfluss spezifischer Supportformen wie der \textit{Brim-Struktur} auf die Detailqualität weniger systematisch untersucht wurde \cite{FFFPolymerReview}.  

Durch die experimentelle Analyse und den Vergleich verschiedener Brim-Parameter leistet dieses Projekt einen Beitrag zum besseren Verständnis der Zusammenhänge zwischen Haftungsstrategien, Druckqualität und Nachbearbeitungsaufwand.  
Die gewonnenen Erkenntnisse können sowohl für Forschung als auch Praxis genutzt werden – etwa zur Entwicklung effizienterer Druckprofile oder zur Verbesserung bestehender Slicer-Algorithmen.  

Für Einsteiger und Hobbyanwender liegt der Nutzen des Projekts in der Entwicklung eines klar strukturierten Workflows, der eine reproduzierbare Vorgehensweise beim Einsatz von Brim-Support-Strukturen ermöglicht.  
Dadurch kann die Fehlerrate bei feinen oder mehrteiligen Druckobjekten reduziert und die allgemeine Druckqualität verbessert werden \cite{Facfox2023, JLC3DP2023}.  

Langfristig besteht zudem ein potenzieller Mehrwert für die Softwareentwicklung:  
Die Ergebnisse könnten als Grundlage für die Integration eines automatisierten Brim-Workflows in Open-Source-Slicer wie \textit{OrcaSlicer} oder in Modellierungsumgebungen wie \textit{Blender} dienen.  
Eine solche Implementierung würde Anwenderinnen und Anwendern eine nachvollziehbare, datenbasierte Entscheidungshilfe bieten und somit zu einer Standardisierung von Haftungs- und Supportstrategien im FDM-Bereich beitragen.
